\documentclass[11pt,oneside]{article} 

%% \titleformat{command}[shape]{format}{label}{sep}{before}[after]

% bug in titlesec:
% https://bugs.launchpad.net/ubuntu/+source/texlive-extra/+bug/1574052
\usepackage{titlesec}
%\usepackage[tracking=true]{microtype}
%\titleformat{\chapter}[display]
%  {\normalfont\huge\bfseries}
%  {\filcenter\underline{\MakeUppercase{\textls[400]{\chaptertitlename}}\ \thechapter}}
%  {20pt}{\Huge}

%\usepackage{epsf}
\usepackage{amsmath}
\usepackage{color}
\usepackage{natbib}
\usepackage{framed}
%\usepackage{cite}
\usepackage{tikz}
\usepackage{tikz-cd}

\RequirePackage{amsmath}
\RequirePackage{amssymb}
\RequirePackage{amsthm}
%\RequirePackage{algorithmic}
%\RequirePackage{algorithm}
%\RequirePackage{theorem}
%\RequirePackage{eucal}
\RequirePackage{color}
\RequirePackage{url}
\RequirePackage{mdwlist}

\RequirePackage[all]{xy}
\CompileMatrices
%\RequirePackage{hyperref}
\RequirePackage{graphicx}
%\RequirePackage[dvips]{geometry}

\usepackage{xcolor}
\usepackage{amsmath,amsfonts,amssymb}
\usepackage{graphicx}
\usepackage[caption=false]{subfig}
\usepackage{enumerate}
\usepackage{mathrsfs}

\definecolor{darkblue}{RGB}{0,0,127} % choose colors
\definecolor{darkgreen}{RGB}{0,150,0}
%\hypersetup{colorlinks, linkcolor=darkblue, citecolor=darkgreen, filecolor=red, urlcolor=blue}
%\hypersetup{pdfauthor={Simon Burton}}
%\hypersetup{pdftitle={Foo Foo}}

\usepackage[normalem]{ulem}

\usepackage{setspace}   %Allows double spacing with the \doublespacing command

\newcommand{\todo}[1]{\ \textcolor{red}{\{#1\}}\ }
\newcommand{\danbrowne}[1]{\vspace{10pt}\noindent\textcolor{red}{{\it #1}}}

\newcommand{\Eref}[1]{(\ref{#1})}
\newcommand{\Fref}[1]{Fig.~\ref{#1}}
%\newcommand{\Aref}[1]{Appendix~\ref{#1}}
\newcommand{\SRef}[1]{section~\ref{#1}}


\def\Complex{\mathbb{C}}
\def\C{\mathbb{C}}
\def\R{\mathbb{R}}
\def\Z{\mathbb{Z}}
%\def\Ham{\mathcal{H}} % meh..
\def\Ham{H}
\def\Pauli{\mathcal{P}}
\def\Spec{\mbox{Spec}}
\def\Proveit{{\it (Proof??)}}
\def\GL{\mathrm{GL}}
\def\half{\frac{1}{2}}
\def\Stab{S}



\newcommand{\ket}[1]{|{#1}\rangle}
\newcommand{\expect}[1]{\langle{#1}\rangle}
\newcommand{\bra}[1]{\langle{#1}|}
\newcommand{\ketbra}[2]{\ket{#1}\!\bra{#2}}
\newcommand{\braket}[2]{\langle{#1}|{#2}\rangle}

\newcommand{\bket}[1]{\bigl|\,{#1}\,\bigr\rangle}
\newcommand{\bbra}[1]{\bigl\langle\,{#1}\,\bigr|}
\newcommand{\bketbra}[2]{\bket{#1}\!\bra{#2}}
\newcommand{\bbraket}[2]{\bigl\langle\,{#1}\,\bigr|\,{#2}\,\bigr\rangle}

\def\Span#1{\langle #1 \rangle}

%\newcommand{\todo}[1]{\textcolor{red}{#1}}

\def\smbox#1{\ \ \mbox{#1}\ \ }



\newcommand{\Field}{\mathcal{F}}
\def\Im{\mathrm{im}}
\def\Ker{\mathrm{ker}}
\def\Dim{\mathrm{dim}}
%\def\euler{\chi}
\def\euler{\mu}


\title{Thesis ``non-abelian quantum codes'' edit summary}
\author{Simon David Burton}
\date{November 2017}

\begin{document}

\maketitle

Comments from Dan Browne are highlighted below in red, followed by 
notes on changes made to the thesis.

The other two reviewers did not require any changes be made to the thesis.

\danbrowne{
I have noted some typos and minor improvements (such as rephrasing and missing references) for the candidate to consider as
annotations in a PDF which I can supply to the candidate.
}


\danbrowne{
Apart from those, I request the following corrections:
}

\danbrowne{
Coherence of the thesis:
Currently the these feels like 4 independent articles without a coherent structure. Furthermore the chapters are missing an
essential conclusions section, summarising the results, critiquing the methods used, connecting the results with related work in the
literature and discussing open questions. This can be remedied by the addition of two short extra chapters and a conclusions
section to each chapter.
}

Done.

\danbrowne{
Specifically, I request:
To add a short pre-introductory chapter (or expanded abstract) motivating the research area, and summarising the key results.
}

Chapter 0 added.

\danbrowne{
Conclusions sections to be added to all chapters. These sections should
1. Summarise the results presented
2. Discuss any approximations made and any arbitrary choices made (e.g. a choice of code or of topological qubit) and
discuss how the results obtained in the specific case may give insight into more general cases.
3. Discuss any recent related work (if any exists)
4. Identify key open questions and next steps, and potential application of the results in the chapter.
5. Link to the next chapter (if appropriate)
}

Done: "Discussion" section added at end of chapters 1-4.

\danbrowne{
To add a brief concluding chapter summarising all results presented in the thesis.
}

Chapter 5 added.

\danbrowne{
I have a small number of technical requests on the chapters to be addressed:
Chapter 1
Although I like the unconventional approach here, make sure that the literature is fully cited where appropriate - e.g. when you first
introduce Kitaev’s topic code and the homological approach to quantum codes, mention him and cite the appropriate literature.
}

Citations added to chapter 1.

\danbrowne{
Chapter 2
This is a strong and very interesting chapter, but the key analytic results are presented as “Fact” boxes without formal proofs.
Although these results are often explained informally in the text, this is not sufficient. Please add formal proofs of Facts 0 to 3. If
you don’t want to break the flow of the text I suggest adding an appendix with these proofs.
}

All results in section 2.6 and 2.8 now have proofs.

\danbrowne{
Chapter 3
Please connect with the content of chapter 4. In particular, either here or in chapter 4, introduce a topological qubit (and its logical
operators) on a toric manifold as considered in chapter 4. There’s a strong disconnect at present that chapter 3 focusses solely on
manifolds with holes and chapter 4 solely on a manifold without holes.
}

More discussion added to chapter 3 and 4.
In particular section 3.5 connects chapter 3 and 4.

\danbrowne{
Chapter 4
Please give more justification of your chosen error model.
Is it the model that one would expect arising from thermal noise?}

added discussion on p75.

\danbrowne{Please explain more clearly how pair creation and curve
diagrams lead to specific measurement outcomes for the observables
measured.}

See example on p76.

\danbrowne{
Please explain in more detail how you simulate the errors,
and, in particular, what the quantity $t_{sim}$ represents.
}
Further discussion added to chapter 4.
In particular, the example on page 76 and associated discussion.

\danbrowne{
In the conclusions section, please discuss the extent
to which this approach to error correction could be modified for error
correction of other forms of topological qubit, such
as other anyon species, and scenarios with multiple logical qubits per surface.
}

Done.

\vspace{20pt}
\hrule
\hrule
\hrule
\vspace{20pt}

Further edits:

{\vspace{20pt}\noindent \bf \underline{Chapter 0.}}

\danbrowne{motivate the research area and summarise key results}
Chapter added.



{\vspace{20pt}\noindent \bf \underline{Chapter 1.}}
% \chapter{Introduction}
% %\chapter{A Homological Perspective on Quantum Codes}
%+\section{Homology of a surface}

\danbrowne{What's this?}
Explain rank-nullity theorem.

%+\section{Classical and quantum codes}

\danbrowne{What is the point of this?}
Show table summarizing linear algebra perspective.

\danbrowne{If this is the first place you introduce a homologically non-trivial operator you need to explain why it corresponds to a non-trivial logical operator.}
See previous point.

\danbrowne{newpage or stars?}
Chapter divided into sections.

%+\section{The energetic viewpoint}
%+\section{Two roads to non-abelian codes}
%+\section{Discussion}

\danbrowne{This is verging on too informal for a thesis. }
informal language removed.

%+\section{Acknowledgement}
{\vspace{20pt}\noindent \bf \underline{Chapter 2.}}
% \chapter{Representations and Spectra of Gauge Code Hamiltonians}
% \section{Introduction}

\danbrowne{This sentence doesn't quite make sense.}
Rewrite sentance.

\danbrowne{Only true for stabiliser hamiltonians?}
Not clear how to respond to this.

\danbrowne{How do you compute this? ZZZ doesn't seem to play a role in generating this orbit. Why not?}
add more details.

\danbrowne{It doesn't seem that it is necessary for this to be abelian via your definition. Are you demanding that it be abelian? If so, you should modify your definition. }
Fix definition of Pauli group presentation.


\danbrowne{What sets the phases of these operators? Are they arbitrarily chosen?}
Comment added.


\danbrowne{There's no unique choice here, so it's a bit misleading to call them "the adjacent operators". }
Comment added.

% \section{Group representations}\label{GroupReps}

\danbrowne{Perhaps mention that omega represents -1 here.  You should also mention that this is not the standard definition of the Pauli group, as you have neglected i.  In particular XY is not hermitian. Does it matter?}
footnote added.

% \section{Applications}

\danbrowne{Can you do this? Please justify it? (Presumably ok since Y only occurs as a tensor pair). } 
footnote added.

% %\section{Representations over the finite field $\Field$}
% \section{Symplectic representations}

\danbrowne{I'd suggest you introduce Ht and Htx,tz more prominently. }
Not clear how to respond to this.

% \section{Gapless 1D models}
% \section{Perron-Frobenius theory}

\danbrowne{You need to make it clearer how much of this is newly defined and how much is developing previous work. Is the Perron-Frobenius construction your own?}
further discussion added.


\danbrowne{This is a key result and you speed through it too quickly. Can you write this as a lemma with a formal proof?}
proofs added.


\danbrowne{Please spell out the argument more clearly. It isn't immediate that this follows from your two facts. }
proofs added.

% \section{The gauge color code model}

\danbrowne{Is this correct? All the operators here are built from faces, not cells. }
Not clear how to fix this missunderstanding .

% \section{The orbigraph}

\danbrowne{Please prove this.}
proofs added.

%-\section{Lie algebra representations}
%+\section{Lie algebra theory}

\danbrowne{a bit too colloquial. Explain what eating means in this context!}
changed "eating" to "consumes"

% %\section{Spectra}
% \section{Numerical results}

\danbrowne{What is a frustrated stabiliser? You haven't explained what this is or how it is relevant.}
defined this.


\danbrowne{Can you confirm something like this numerically? Or is a weaker statement true?}
clarifying comment added.


\danbrowne{Does this mean that small codes are not representative of generic codes?}
further discussion added.


\danbrowne{Very nice results. Why did you stop at n=175? What computational resources were required to achieve that?}
further discussion added.

% \section{Cheeger cuts}
%+\section{Discussion}

\danbrowne{Needs a concluding section summarising results and discussing open questions and potential research directions. }
added.

% %\section{Outlook}
{\vspace{20pt}\noindent \bf \underline{Chapter 3.}}
% \chapter{A Short Guide to Anyons and Modular Functors}
% %\section{A TQFT warmup}
% %\section{Introduction}
% \section{Overview}

\danbrowne{This justification seems a little vague. Could you make it better justified by making contact with the literature on superselection rules?}
comment removed.

% \section{Topological Exchange Statistics}
% \section{Modular Functors}\label{ModularFunctors}

\danbrowne{Why is this unitarity?}
clarifying comment added.


\danbrowne{What does POP stand for?}
clarifying comment added.


\danbrowne{Is this the origin of the superselection rule in the theory?}
Not clear how to respond to this.


\danbrowne{Illustrate this with a figure?}
No figure added...


\danbrowne{A citation here to ribbons used in skein theory. }
citation added.

% \section{Refactoring theorem}\label{RefactoringTheorem}
%+\section{Discussion}

\danbrowne{Needs summary and conclusions and link to next chapter. }
summary added.

{\vspace{20pt}\noindent \bf \underline{Chapter 4.}}
% \chapter{Error Correction in a Non-Abelian Topologically Ordered System}
% \section{Fibonacci anyons}
% \section{Physical model}

\danbrowne{Clarify this with a figure. In chapter 3 you have only considered surfaces with boundary. How is a logical qubit encoded on a surface with no boundary?  Why is there a single qubit here (and not a higher number)?  How are logical operations implemented in this model? Important to introduce the logical operators that will form your logical errors.  Why did you choose this qubit encoding?}
further discussion added.


\danbrowne{Add a link to refer back to the section in chapter 3 where you introduced observables. }
footnote added.


\danbrowne{What's the (physical) justification for this error model? Would thermal excitations manifest themselves like this?}
Added discussion of Hamiltonian, and other error models.


\danbrowne{Have you explained this measurement rule anywhere? In particular, how is the outcome of the measurement derived from the curve? Is this measurement deterministic or non-deterministic?}
Detailed example added on p76.


\danbrowne{Why does homology play a role here? Don't forget to introduce the logical operators of this qubit. }
removed mention of homology. We do not introduce the logical operators, this would require further research.

% \section{Decoding algorithm}
% \section{Simulation of the quantum system}
% \section{Numerical results}
% %\section{Computation of homologically non-trivial operators}\label{s:homnontrivial}
%+\section{Discussion}

{\vspace{20pt}\noindent \bf \underline{Chapter 5.}}
%+\chapter{Conclusion}

\danbrowne{brief chapter summarising all results}
Added.

% %\chapter{First Appendix}

\end{document}


