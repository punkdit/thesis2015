\documentclass[11pt,oneside]{article} 

%% \titleformat{command}[shape]{format}{label}{sep}{before}[after]

% bug in titlesec:
% https://bugs.launchpad.net/ubuntu/+source/texlive-extra/+bug/1574052
\usepackage{titlesec}
%\usepackage[tracking=true]{microtype}
%\titleformat{\chapter}[display]
%  {\normalfont\huge\bfseries}
%  {\filcenter\underline{\MakeUppercase{\textls[400]{\chaptertitlename}}\ \thechapter}}
%  {20pt}{\Huge}

%\usepackage{epsf}
\usepackage{amsmath}
\usepackage{color}
\usepackage{natbib}
\usepackage{framed}
%\usepackage{cite}
\usepackage{tikz}
\usepackage{tikz-cd}

\RequirePackage{amsmath}
\RequirePackage{amssymb}
\RequirePackage{amsthm}
%\RequirePackage{algorithmic}
%\RequirePackage{algorithm}
%\RequirePackage{theorem}
%\RequirePackage{eucal}
\RequirePackage{color}
\RequirePackage{url}
\RequirePackage{mdwlist}

\RequirePackage[all]{xy}
\CompileMatrices
%\RequirePackage{hyperref}
\RequirePackage{graphicx}
%\RequirePackage[dvips]{geometry}

\usepackage{xcolor}
\usepackage{amsmath,amsfonts,amssymb}
\usepackage{graphicx}
\usepackage[caption=false]{subfig}
\usepackage{enumerate}
\usepackage{mathrsfs}

\definecolor{darkblue}{RGB}{0,0,127} % choose colors
\definecolor{darkgreen}{RGB}{0,150,0}
%\hypersetup{colorlinks, linkcolor=darkblue, citecolor=darkgreen, filecolor=red, urlcolor=blue}
%\hypersetup{pdfauthor={Simon Burton}}
%\hypersetup{pdftitle={Foo Foo}}

\usepackage[normalem]{ulem}

\usepackage{setspace}   %Allows double spacing with the \doublespacing command

\newcommand{\todo}[1]{\ \textcolor{red}{\{#1\}}\ }
\newcommand{\danbrowne}[1]{\vspace{10pt}\noindent\textcolor{red}{{\it #1}}}

\newcommand{\Eref}[1]{(\ref{#1})}
\newcommand{\Fref}[1]{Fig.~\ref{#1}}
%\newcommand{\Aref}[1]{Appendix~\ref{#1}}
\newcommand{\SRef}[1]{section~\ref{#1}}


\def\Complex{\mathbb{C}}
\def\C{\mathbb{C}}
\def\R{\mathbb{R}}
\def\Z{\mathbb{Z}}
%\def\Ham{\mathcal{H}} % meh..
\def\Ham{H}
\def\Pauli{\mathcal{P}}
\def\Spec{\mbox{Spec}}
\def\Proveit{{\it (Proof??)}}
\def\GL{\mathrm{GL}}
\def\half{\frac{1}{2}}
\def\Stab{S}



\newcommand{\ket}[1]{|{#1}\rangle}
\newcommand{\expect}[1]{\langle{#1}\rangle}
\newcommand{\bra}[1]{\langle{#1}|}
\newcommand{\ketbra}[2]{\ket{#1}\!\bra{#2}}
\newcommand{\braket}[2]{\langle{#1}|{#2}\rangle}

\newcommand{\bket}[1]{\bigl|\,{#1}\,\bigr\rangle}
\newcommand{\bbra}[1]{\bigl\langle\,{#1}\,\bigr|}
\newcommand{\bketbra}[2]{\bket{#1}\!\bra{#2}}
\newcommand{\bbraket}[2]{\bigl\langle\,{#1}\,\bigr|\,{#2}\,\bigr\rangle}

\def\Span#1{\langle #1 \rangle}

%\newcommand{\todo}[1]{\textcolor{red}{#1}}

\def\smbox#1{\ \ \mbox{#1}\ \ }



\newcommand{\Field}{\mathcal{F}}
\def\Im{\mathrm{im}}
\def\Ker{\mathrm{ker}}
\def\Dim{\mathrm{dim}}
%\def\euler{\chi}
\def\euler{\mu}


\title{Thesis ``non-abelian quantum codes'' edit summary}
\author{Simon David Burton}
\date{November 2017}

\begin{document}

\maketitle

Comments from Dan Browne are highlighted below in red, followed by 
notes on changes made to the thesis.

The other two reviewers did not require any changes be made to the thesis.

\danbrowne{
I have noted some typos and minor improvements (such as rephrasing and missing references) for the candidate to consider as
annotations in a PDF which I can supply to the candidate.
}


\danbrowne{
Apart from those, I request the following corrections:
}

\danbrowne{
Coherence of the thesis:
Currently the these feels like 4 independent articles without a coherent structure. Furthermore the chapters are missing an
essential conclusions section, summarising the results, critiquing the methods used, connecting the results with related work in the
literature and discussing open questions. This can be remedied by the addition of two short extra chapters and a conclusions
section to each chapter.
}

Done.

\danbrowne{
Specifically, I request:
To add a short pre-introductory chapter (or expanded abstract) motivating the research area, and summarising the key results.
}

Chapter 0 added.

\danbrowne{
Conclusions sections to be added to all chapters. These sections should
1. Summarise the results presented
2. Discuss any approximations made and any arbitrary choices made (e.g. a choice of code or of topological qubit) and
discuss how the results obtained in the specific case may give insight into more general cases.
3. Discuss any recent related work (if any exists)
4. Identify key open questions and next steps, and potential application of the results in the chapter.
5. Link to the next chapter (if appropriate)
}

Done: "Discussion" section added at end of chapters 1-5.

\danbrowne{
To add a brief concluding chapter summarising all results presented in the thesis.
}

Chapter 6 added.

\danbrowne{
I have a small number of technical requests on the chapters to be addressed:
Chapter 1
Although I like the unconventional approach here, make sure that the literature is fully cited where appropriate - e.g. when you first
introduce Kitaev’s topic code and the homological approach to quantum codes, mention him and cite the appropriate literature.
}

Citations added to chapter 1.

\danbrowne{
Chapter 2
This is a strong and very interesting chapter, but the key analytic results are presented as “Fact” boxes without formal proofs.
Although these results are often explained informally in the text, this is not sufficient. Please add formal proofs of Facts 0 to 3. If
you don’t want to break the flow of the text I suggest adding an appendix with these proofs.
}

All results in section 2.6 and 2.8 now have proofs.

\danbrowne{
Chapter 3
Please connect with the content of chapter 4. In particular, either here or in chapter 4, introduce a topological qubit (and its logical
operators) on a toric manifold as considered in chapter 4. There’s a strong disconnect at present that chapter 3 focusses solely on
manifolds with holes and chapter 4 solely on a manifold without holes.
}

More discussion added to chapter 3 and 4.
In particular section 3.5 connects chapter 3 and 4.

\danbrowne{
Chapter 4
Please give more justification of your chosen error model.
Is it the model that one would expect arising from thermal noise?}

added discussion on p75.

\danbrowne{Please explain more clearly how pair creation and curve
diagrams lead to specific measurement outcomes for the observables
measured.}

See example on p76.

\danbrowne{
Please explain in more detail how you simulate the errors,
and, in particular, what the quantity $t_{sim}$ represents.
}
Further discussion added to chapter 4.
In particular, the example on page 76 and associated discussion.

\danbrowne{
In the conclusions section, please discuss the extent
to which this approach to error correction could be modified for error
correction of other forms of topological qubit, such
as other anyon species, and scenarios with multiple logical qubits per surface.
}

Done.

\vspace{10pt}
\hrule
\hrule
\hrule




\end{document}



