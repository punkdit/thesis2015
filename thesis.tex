\documentclass[11pt,a4paper]{article}
%\documentclass[11pt, twocolumn]{article}

%\usepackage[paper=a4paper,dvips,top=1.5cm,left=1.5cm,right=1.5cm,foot=1cm,bottom=1.5cm]{geometry}
\usepackage[paper=a4paper,dvips,top=3.0cm,left=1.5cm,right=1.5cm,foot=1cm,bottom=1.5cm]{geometry}

%\usepackage{epsf}
\usepackage{amsmath}
\usepackage{color}
\usepackage{natbib}
%\usepackage{cite}

\RequirePackage{amsmath}
\RequirePackage{amssymb}
\RequirePackage{amsthm}
%\RequirePackage{algorithmic}
%\RequirePackage{algorithm}
%\RequirePackage{theorem}
%\RequirePackage{eucal}
\RequirePackage{color}
\RequirePackage{url}
\RequirePackage{mdwlist}

\RequirePackage[all]{xy}
\CompileMatrices
\RequirePackage{hyperref}
\RequirePackage{graphicx}
\RequirePackage[dvips]{geometry}


\begin{document}

\title{Abelian and Non-Abelian Quantum Codes}

\author{Simon Burton}

\maketitle

%{\it $^1$ Centre for Engineered Quantum Systems, School of Physics, The University of Sydney, Sydney, Australia}


\def\Complex {C}
\def\tensor{\otimes}
\def\Tensor{\bigotimes}
\def\bra #1{\langle #1|}
\def\ket #1{|#1\rangle}
\def\braket #1#2{\langle #1|#2 \rangle}

%\def\Set{\widetilde{\text{Set}}}
%\def\Top{\widetilde{\text{Top}}}
%\def\Vec{\widetilde{\text{Vec}}}
%\def\Chain{\widetilde{\text{Chain}}}

%\def\ker{\text{ker}}
%\def\coker{\text{coker}}
%\def\im{\text{im}}

%\def\H{\mathcal{H}}
%\def\H{H}
%\def\S{S}
\def\mathZ{\mathbb{Z}}
\def\mathR{\mathbb{R}}

%\def\nin{\not\in}

%\def\heading #1{\noindent\underline{\Large\bf #1}}
\def\heading #1{\vskip 20pt \noindent\underline{\large \bf #1}\vskip 5pt}

\def\important #1{\underline{\bf #1}}


% ----------------------------------------------------------------------------

\heading{Abstract}

Technical contribution: 
(1) classification theorem
(2) discretization of curve diagrams
(3) simulation

% ----------------------------------------------------------------------------

\heading{Introduction}

Fibonacci hard


the dimensionality of the state space
of anyons grows as $d^n$.
For fibonacci anyons $d\simeq 1.6$ and
it is known that braiding generates a dense
set of unitaries on the state space, so we
only expect to be able to simulate for small $n.$

Below bond-percolation threshold we expect to see clusters
of charges of size $O(log(n))$ with variance $O(1)$ \cite{Bazant00}.

DIAGRAM

% ----------------------------------------------------------------------------

\heading{Curve diagrams}

The disc $D = \{ x\in \mathR^2\ s.t.\ |x|\leq 1 \} $,
has boundary $\partial D = \{ x\in \mathR^2\ s.t.\ |x|=1 \} $.

Given a finite set $ Q_n \subset D-\partial D$,
we denote $D_n$ as the pair $(D, Q_n).$
%Topologically, this definition does not depend on
%anything but the cardinality of $Q_n.$

We define the mapping class group of $D_n$,
$MCG(D_n),$ as the set of homeomorphisms $D\to D$ that
restrict to a permutation on $Q_n,$ modulo isotopy that
fixes $Q_n.$
Such an equivelance class
of homeomorphisms will be denoted as $\phi:D_n\to D_n.$

We can think of $D_n$ as the unit disc with $n$ holes,
and the mapping class group as the set of homeomorphisms on
this space, modulo isotopy.

The line $L = \{ x\in \mathR\ s.t.\ |x|\leq 1 \} $,
and a finite set $R_n\subset L$,
$L_n = (L, R_n).$
By a slight abuse of notation, we enumerate the points in $R_n$ 
in a monotonically increasing order,
and understand such notation as $[i, i+1]$ to mean the closed interval in $L$
with (consecutive) endpoints $i, i+1 \in R_n.$

We define a {\it curve diagram} as an embedding $f : L\to D$ that
restricts to a bijection $R_n\to Q_n,$ modulo
isotopy.
This will be denoted as $f : L_n\to D_n.$
We can generalize this to $f : L_m\to D_n$ with $m\leq n$
to mean a map $L\to D$ that restricts to an injection $R_m\to Q_n,$
modulo isotopy.

\begin{center}
\includegraphics[width=0.4\textwidth]{curve-diagram.eps}
\end{center}

XXX $\partial L\to \partial D$ XXX

Given two disjoint curve diagrams $f:L_m\to D_n$ and $f':L_{m'}\to D_n$
we define their sum  $f+f':L_{m+m'}\to D_n$ ...

% ----------------------------------------------------------------------------

\heading{Group action}

The mapping class group $MCG(D_n)$ acts on curve diagrams $f : L_n\to D_n$
via post-composition (left multiplication).
This action is transitive and XXX so we can identify an element of
$MCG(D_n)$ by its action on any curve $f:L_n\to D_n.$
See \cite{Dehornoy02}, chapter 6.
XXX $\partial L\to \partial D$ XXX


%\heading{Braid group representation}

We single out specific ``braid'' elements of $MCG(D_n)$.
These act on the image of a pair-of-pants embedding: $g:D_2\to D_n.$

DIAGRAM

The {\it braid group} on $n$ strands is the group $B_n$ generated by $n-1$ generators
$\sigma_1, \sigma_2, ... \sigma_{n-1}$ with the following relations:
    $$ \sigma_i \sigma_j = \sigma_j \sigma_i \ \text{when}\ |i-j| > 2, $$
    $$ \sigma_i \sigma_{i+1} \sigma_i =  \sigma_{i+1} \sigma_i \sigma_{i+1}.$$


Here we verify the braid relations directly:

\begin{center}
%\includegraphics[width=0.5\textwidth]{mypicture.png}
\includegraphics[width=0.5\textwidth]{curve-braid.eps}
\end{center}


\important{Definition:} by a {\it half-twist} on a curve
diagram $f:L_n\to D_n$ at $i$ we mean an element of $MCG(D_n)$
that can be represented by a half-twist that acts
on a small neighbourhood of $f([i, i+1]).$
We denote a clockwise half-twist by $b(i, f):$

\begin{center}
\includegraphics[width=0.2\textwidth]{halftwist-1.eps}
\includegraphics[width=0.15\textwidth]{halftwist-2.eps}
\end{center}

Each half-twist corresponds to a braid action of two neighbouring
holes on $f$.
\footnote{See \cite{Kassel10}, Section 1.6.2, for more details on half-twists.}

% ----------------------------------------------------------------------------

\heading{Anyons}


We measure charge contained within a disc. % region, circle
Measurement outcomes belong to a set of charges $\Lambda.$
Commuting measurements are either disjoint discs or
one wholly contained within the other.
So (proof?) we get a basis for the Hilbert space of our system from
a pair-of-pants (POP) decomposition:

%\begin{center}
%(i)
%\includegraphics[width=0.2\textwidth]{POP-1.eps}
%\hskip 10pt
%(ii)
%\includegraphics[width=0.2\textwidth]{POP-2.eps}
%\hskip 10pt
%(iii)
%\includegraphics[width=0.14\textwidth]{POP-3.eps}
%\end{center}

\begin{center}
\includegraphics[width=0.2\textwidth]{POP-1.eps}
\end{center}

The feet of the POP decomposition form the $n$ holes $Q_n$ in the disc $D$.
\footnote{Every surface that is not a sphere, torus, disc or annulus, has
a POP decomposition, see \cite{Ivanov01}, Theorem 2.4.A.}

We say that a POP decomposition of $D_n$
{\it admits} a curve $f:L_n\to D_n$
if (there is a curve in the isotopy class of)
the curve performs a depth first traversal of the POP.
This means that if the curve visits the trunk of a POP
then it visits all the legs of the POP before leaving that POP.

\begin{center}
\includegraphics[width=0.2\textwidth]{POP-2.eps}
\end{center}

Conversely, given a curve $f:L_n\to D_n$
and a binary rooted tree $t$,
we can recover a unique POP that admits $f$.
The above POP would be represented by the following tree:

\begin{center}
\includegraphics[width=0.14\textwidth]{POP-3.eps}
\end{center}

(proof: use small neighbourhoods of the curve)

\important{F-moves:} given a nested POP and admissable $f$,
the F-move is the change of basis as follows:

\begin{center}
\includegraphics[width=0.3\textwidth]{POP-4.eps}
\end{center}

XXX F-moves can be nested

\important{R-moves:} we examine the action of $MCG(D_n)$
on our state space. Given a single POP and curve $f:L_2\to D_n$,
the half-twist on $f$ acts on $\psi$ as $R\psi$.

DIAGRAM?

This together with f-moves gives all the half-twists on
the curve $f$. Therefore, we may work solely with curve
diagrams, forgetting the underlying POP (tree) structure.

We have a (linear) representation of the braid group:

    $$ \kappa : B_n \to GL(F_n).$$

Where $F_n$ is the Hilbert state space of $n$ Fibonacci anyons.
%with trivial total charge.

See \cite{Pfeifer12, Pfeifer14}


% ----------------------------------------------------------------------------

\heading{Half-twist factorization of $MCG(D_n)$}

\important{Definition:} by a {\it sequence of half-twists} on a curve
diagram $f:L_n\to D_n$ we mean a
sequence of curve diagrams $f_k: L_n\to D_n$, and a sequence of half-twists:

        $$ b(i_1, f_1),\ b(i_2, f_2),\ ...\ b(i_N, f_N) $$

such that $f_1=f,$ and

        $$ f_{k+1} = b(i_k, f_k) f_k,\ \  \text{} 1\leq k<N.$$


\important{Problem:}
Given curve diagrams $f:L_n\to D_n$ and $g:L_2\to D_n$
find $\phi\in MCG(D_n)$ as a product of
a sequence of half-twists such that
$g = \phi f i,$ where $i$ is an inclusion of $L_2$ into $L_n:$

\begin{center}
\includegraphics{halftwist-factor.eps}
\end{center}

The idea is that we would like to apply half-twists to a curve diagram
until the two holes indicated by $g$ become neighbours.

\important{Examples:}

In this example we use an inverse half-twist about the segment $[a, a+1]:$

\begin{center}
\includegraphics[width=0.6\textwidth]{example-problem-1.eps}
\end{center}


\important{Solution:}

1) decompose $f$ along (finitely many) intersections
with $g.$

\begin{center}
\includegraphics[width=0.3\textwidth]{snake-decompose.eps}
\end{center}


2) on each component $k$:

\begin{center}
\includegraphics[width=0.3\textwidth]{snake-component.eps}
\end{center}

let $h_k = g_k + f_k.$
This is a simple closed curve in $D$ which splits $D$
into two regions, the inside (simply connected) and the
outside.
We find three basic cases (there are more including reflections
and rotations of these), which we can distinguish between
using the winding number and whether the end points
of $f$ are on the inside or outside of $h_k$:

\begin{center}
\includegraphics[width=0.5\textwidth]{snake-cases.eps}
\end{center}

XXX

% ----------------------------------------------------------------------------

\heading{Planar diagrams}

We aim to discretize the problem for software implementation.
Our first task is to characterize what happens to a curve
if we ``forget'' any braiding inside a sub-disc.
We adapt the Abramsky combinatorial description 
of the Temperly-Lieb algebra to our needs
(see \cite{Abramsky08}, section 6.) %Proposition 1.6).

The set of {\it planar diagrams} is defined as:

    $$ P = \{ f(L)\cap E / \sim \ \text{s.t.}\ f:L\to D \}$$

where $f:L\to D$ is a continuous function 
whose image $f(L)$ intersects a disc $E\subset D$ in a finite number of components,
and the equivalence relation $\sim$ consists of 
isotopies that fix $\partial E.$ % Local (relative?) isotopies
Each component of $f(L)\cap E$ we call a {\it strand.}

Elements of $P$ have a simple combinatorial description 
as a finite sequence (a word) consisting of
the four labels {\bf (, ), H, T} such that 
the parentheses are balanced.
To construct such a word, begin at a point on $\partial E$, and proceeding
clockwise: visiting a strand for the first time produces a {\bf ( } label,
and visiting a strand for the second time produces a {\bf ) } label.
The two labels {\bf H} and {\bf T} specify the
head and tail endpoints of $f$, each of which appears at most once in the word.

For example, the following planar diagram can be represented by
the word {\bf(())H}:

\begin{center}
\includegraphics[width=0.5\textwidth]{planar.eps}
\end{center}


% ----------------------------------------------------------------------------

\heading{Tiling}

%Now we consider a tiling of the disc into subdiscs that are disjoint on
%their interiors.
We now consider a cellulation of a disc.
Concretely, we choose a regular $l$ by $l$ square tiling of a square region.
The holes $Q_n$ will always be contained within the interior of the
tiles.

\important{Definition:} the discrete curve space $C_n^{(l)}$ consists of
curves $L_n\to D_n$ modulo isotopy that fixes the boundary of
every tile.

Combinatorialy, we store an element of $C_n^{(l)}$ by patching
together words for planar diagrams in a compatible way: 
neighbouring tiles must agree on the strands that they share.

\begin{center}
\includegraphics[width=0.3\textwidth]{discrete-curve.eps}
\end{center}

% ----------------------------------------------------------------------------

\heading{Simulation and noise model}

\begin{center}
(i)
\includegraphics[width=0.26\textwidth]{pair-create.eps}
\hskip 10pt
(ii)
\includegraphics[width=0.26\textwidth]{syndrome-1.eps}
\hskip 10pt
(iii)
\includegraphics[width=0.26\textwidth]{syndrome-2.eps}
\end{center}

We Maintain a collection of disjoint curves.

(i) Poisson process, pair creation $t_{\mathrm{sim}}$.

(ii) join fusion trees (curves)
that participate in the same site (tile).
(iii) measure charge on each site and leave result on
one of the strands.
These charge measurements give the initial syndrome for the
error correction procedure.

Multi-round, clustering of charges, \cite{Brell13} \cite{Bravyi13}

Torus:
Fail on big braid

% ----------------------------------------------------------------------------

\heading{Numerical results}

\begin{center}
\includegraphics[width=0.5\textwidth]{threshold-graph.eps}
\end{center}

Threshold appears at $t_{\mathrm{sim}}\simeq 0.124 \pm 0.004$

Plot of computational power...

% ----------------------------------------------------------------------------

\heading{Conclusion/Discussion}

% ----------------------------------------------------------------------------

\important{Acknowledgements}

Yo to my bro's and homies

% ----------------------------------------------------------------------------

%Given a topological space $X$ we define the {\it mapping class group} of $X$ as
%the set of homeomorphisms $f:X\to X$ modulo isotopy
%
%    % $$ MCG(X) := \{ f : X \to X \text{s.t.} f \text{is homeomorphism} \} / \sim_{\text{iso}} $$
%    $$ MCG(X) := \{ f : X \to X \} / \sim_{\text{iso}}.$$

%Given a topological manifold $M$ we define the {\it mapping class group} of $M$ as
%the set of homeomorphisms $f:M\to M$ modulo isotopy
%
%    $$ MCG(M) := \{ f : M \to M \} / \sim_{\text{iso}}.$$
%
%If $M$ has boundary $\partial M$ we require the homeomorphisms to
%be the identity on $\partial M$.
%
%In particular we are interested in
% the $n$-punctured disc.
%This is a disc $D = \{ x\in \mathR^2 s.t. |x|\leq 1 \} $
%with $n$ points removed:
%
%    $$ D_n := D - Q_n,$$
%
%where $Q_n$ is some finite subset of the interior of $D$.
%
%%topologically this is a sphere with $n+1$ holes in it.
%
%It is a theorem that the group $MCG(D_n)$ is isomorphic to $B_n.$
%(See \cite{Kassel10}, chapter 1.)
%
%We fix $Q_n$ to be the points $\{(0, i/(n+1)) for i=1,...n\}$
%then we define the line $L = \{(0, x) s.t. |x|\leq 1\}.$
%Given a representative $f$ of an element of $MCG(D_n)$
%we define the
%curve diagram as the
%image of $L$ under $f$.
%
%It is a theorem (see \cite{Dehornoy02}) that such curve diagrams
%correspond uniquely to elements of the mapping class group, and
%hence to the braid group $B_n$.

% ----------------------------------------------------------------------------

%\heading{Pair-of-pants decompositions}

%In an abelian theory we measure charge in regions bounded by curves, ie. elements of
%the first homology group $H_1(D_n).$


%http://arxiv.org/pdf/1002.2816.pdf
%http://www.math.ucsb.edu/~zhenghwa/data/course/lecture_notes/Chap2.pdf
%http://physics.stackexchange.com/questions/93183/about-the-atiyah-segal-axioms-on-topological-quantum-field-theory
%http://mathoverflow.net/questions/359/a-reading-list-for-topological-quantum-field-theory
%http://projecteuclid.org/download/pdf_1/euclid.cmp/1104178138
%http://arxiv.org/pdf/math/9912085v1.pdf



% ----------------------------------------------------------------------------


\bibliography{refs}{}
\bibliographystyle{abbrv}

\end{document}


