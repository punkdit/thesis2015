%thesis.tex 
%Model LaTeX file for Ph.D. thesis at the 
%School of Mathematics, University of Edinburgh

\documentclass[11pt,twoside,openright]{report} 

%\usepackage{epsf}
\usepackage{amsmath}
\usepackage{color}
\usepackage{natbib}
\usepackage{framed}
%\usepackage{cite}
\usepackage{tikz}
\usepackage{tikz-cd}

\RequirePackage{amsmath}
\RequirePackage{amssymb}
\RequirePackage{amsthm}
%\RequirePackage{algorithmic}
%\RequirePackage{algorithm}
%\RequirePackage{theorem}
%\RequirePackage{eucal}
\RequirePackage{color}
\RequirePackage{url}
\RequirePackage{mdwlist}

\RequirePackage[all]{xy}
\CompileMatrices
%\RequirePackage{hyperref}
\RequirePackage{graphicx}
%\RequirePackage[dvips]{geometry}

\usepackage{xcolor}
\usepackage{amsmath,amsfonts,amssymb}
\usepackage{graphicx}
\usepackage[caption=false]{subfig}
\usepackage{enumerate}
\usepackage{mathrsfs}


%\usepackage{epstopdf} % to include .eps graphics files with pdfLaTeX

\usepackage[pdfpagelabels,pdftex,bookmarks,breaklinks]{hyperref}
\definecolor{darkblue}{RGB}{0,0,127} % choose colors
\definecolor{darkgreen}{RGB}{0,150,0}
%\hypersetup{colorlinks, linkcolor=darkblue, citecolor=darkgreen, filecolor=red, urlcolor=blue}
%\hypersetup{pdfauthor={Simon Burton}}
%\hypersetup{pdftitle={Foo Foo}}

\usepackage[normalem]{ulem}

\usepackage{setspace}   %Allows double spacing with the \doublespacing command

\newcommand{\todo}[1]{\textcolor{red}{#1}}

%\newcommand{\Eref}[1]{Eq.~(\ref{#1})}
%\newcommand{\Fref}[1]{Fig.~\ref{#1}}
%\newcommand{\Aref}[1]{Appendix~\ref{#1}}
\newcommand{\SRef}[1]{section~\ref{#1}}




\title{Exploring Abelian and Non-Abelian Quantum Codes}
\author{Simon David Burton}
\date{2016}

\usepackage[phd]{edmaths}

\begin{document}

\maketitle

%\doublespacing
%\onehalfspacing

\declaration

%\dedication{To X Y Z}

\tableofcontents

\chapter{Introduction}

We begin our journey with a consideration of ``size'', or ``counting''.
To count the size of something $A$ we write $\mu(A).$
Size is \emph{additive} in the sense of 
$\mu(A\cup B) = \mu(A) + \mu(B)$ except that
$A$ and $B$ may have intersection.
In this case we would have counted the 
size of the intersection twice and so we modify this formula as
$$
    \mu(A\cup B) = \mu(A) + \mu(B) - \mu(A\cap B).
$$
We can continue this idea to find the
size of the union of three pieces
\begin{center}
\includegraphics{pic-ABC.pdf}
\end{center}
In this case the formula reads:
\begin{align*}
\mu(A\cup B\cup C) = \ &\mu(A) + \mu(B) + \mu(C)  \\
                     &- \mu(A\cap B) - \mu(A\cap C) - \mu(B\cap C) \\
                     &+ \mu(A\cap B \cap C).
\ \ \ \ \ \ \ \ \ \ \ \ \ \ \ \ \ \ \ \ \ \ \ \ \ \ \ \ \ \ \ \ \ \ \ \ \ \mbox{(*)}
\end{align*}
The point here is the alternating signs:
each time we try to count a size we overcount by
one intersections worth, subtracting those intersections
goes too far in the opposite direction and so we need
to add intersections of intersections, and so on.

We now wish to apply this idea to count the
size of a sphere. The trick here is to tile the
sphere with the faces of a cube:
\begin{center}
\includegraphics{pic-cube.pdf}
\end{center}
So we have six tiles and one might suggest that 
$\mu(S^2)=6$ but these are closed tiles, so they
intersect on their edges, of which we have 12.
But these edges intersect at vertices and there are
8 of these. 
We extend the above formula (*) to calculate:
$$
    \mu(S^2) = 6 - 12 + 8 = 2.
$$
So the sphere has ``size'' two!
The magic here is that any other convex polyhedron would give
the same answer of two.
This, of course, is known as the \emph{Euler characteristic},
and for a sphere this is indeed two.



%\chapter{A Homological Perspective on Quantum Codes}
%
%
\def\Complex {C}
\def\tensor{\otimes}
\def\Tensor{\bigotimes}

\def\Stab{\text{\tt Stab}}
\def\Logical{\text{\tt Logical}}
\def\Error{\text{\tt Error}}
\def\Guage{\text{\tt Guage}}

\def\Set{\widetilde{\text{Set}}}
\def\Top{\widetilde{\text{Top}}}
\def\Vec{\widetilde{\text{Vec}}}
\def\Chain{\widetilde{\text{Chain}}}

\def\ker{\text{ker}}
\def\coker{\text{coker}}
\def\im{\text{im}}

%\def\H{\mathcal{H}}
\def\H{H}
\def\S{S}
\def\Z{\mathbb{Z}}

\def\nin{\not\in}

%\begin{abstract}
%We adopt the category of length-3 chain complexes over
%$\Z_2$ as the canonical definition of the category of
%(CSS) stabilizer codes. 
%In this category we show how
%%RG comes from a retract,
%the tensor product gives the generalized hypergraph 
%product~\cite{Tillich09}~\cite{Freedman13}
%and welding~\cite{Michnicki12} comes from a pushout diagram.
%Motivation is provided by showing these constructions
%in more well known categories.
%Finally we show how to generalize the category to
%handle non-CSS codes.
%\end{abstract}

% ----------------------------------------------------------------------------
%

\section{Introduction}

%For the sake of simplicity we consider CSS Codes.

%We first introduce the abstract notion of a chain complex.
%This is the central concern of the study of homological
%algebra. 
%%Next we show how topological spaces give rise to chain
%%complexes.
%Historically, these constructions came from the study
%of invariants of topological spaces, but have since found
%application in areas of algebra (commutative rings, etc.) disjoint from topology.
%%Then we show how to store topological information in a chain complex
%%and then codes are chains
%Classical linear codes and quantum stabilizer codes
%can also be seen as homological objects.
%The intersection of these two areas give topological
%codes (\cite{Dennis01}, \cite{Freedman02}, \cite{Bombin06}).
%But we also find application
%of abstract homological constructs to building codes.

%In defense of arrows
%Physicists tend to ignore domain and range
%Category theory is a kind of type theory, akin to
%dimensional analysis, is group theory  (free groups)
%where any type is composable (product of two values),
%but for example
%tensor products and contractions force one
%to keep track of composability (hense tensor networks;
%arrow diagrams for physicists.)
%Equations become commuting diagrams.
%Dimensionless constants.. what does the type theory
%say of itself?
%Sometimes dimensional analysis gets you to the answer
%even though you don't know what you are doing. It's smart
%enough to do it for you. Similarly with categories, sometimes
%it's enough to just compose arrows together in some obvious
%way.
% Also compare: Heisenberg Vs Schrodinger picture.
% Heisenberg is the arrows (operators) to Schrodinger's elements (states).

% Arrows help us keep track of which matrices we
% can multiply together!

% QIT is infused with ad-hoc methods. The author feels
% that a more abstract viewpoint will help to corral 
% the explosion of these ad-hoc constructions. [ref D.P.]

% ----------------------------------------------------------------
%
%
%
% ----------------------------------------------------------------


\section{Symplectic structure of stabilizer codes}

We work with vector spaces over the field $\mathbb{Z}_2.$

A quantum CSS code is given by two
parity check matrices $\S_z$ and $\S_x.$

Such a code will be called {\it regular}
when the parity checks have full rank.

Given a regular code, we can
a symplectic structure is any
solution to the following (block)
matrix equation:

$$
\left(
\begin{array}{c}
L_z \\
\S_z \\
T_z \\
\end{array}
\right)\left(
\begin{array}{c}
L_x \\
T_x \\
\S_x \\
\end{array}
\right)^\top = I,
$$

where $I$ denotes the appropriate
identity matrix, and the small $T$
is matrix transposition.

In general this is a non-linear
equation because of the presense
of the quadratic term: $T_zT_x^\top=0.$

To construct solutions given
$\S_z$ and $S_x$ we proceed as follows:

{\it (1)} 
Find $L_z$.
The rows of $L_z$ lie in the kernel of $\S_x$,
chosen to be (arbitrary) elements of the cosets
of $\S_z^\top$. Ie. the rows of $L_z$ span $\ker(\S_x)/Im(\S_z^t)$.

{\it (2)}
Find $L_x$.
We first repeat step {\it (1)} on the dual code ($\S_x$ and $\S_z$
swapped) to find $L_x'$.
We look for $L_x$ such that $L_zL_x^\top=I$
knowing that the rows of $L_x$ lie in the
span of $L_x'.$ Ie.  $ L_x = AL_x'$ for some $A$.
Now solve $L_zL_x^\top A^\top = I$ for $A.$

{\it (3)}
Find $T_z$. This will be a solution
of the linear system:

\begin{align*}
    (*)\ \ \S_x T_z^\top &= I \cr
    L_x T_z^\top &= 0.\cr
\end{align*}

The solution space has kernel spanned by
the rows of $\S_z.$

{\it (4)}
$T_z$ is now a solution of the linear system:
\begin{align*}
    \S_z T_x^\top &= I \cr
    L_z T_x^\top &= 0\cr
    T_z T_x^\top &= 0.\cr
\end{align*}

From $(*)$ above, we know that $T_z$ has
full rank, and so this system has a unique
solution $T_x.$




% ----------------------------------------------------------------
%
%  The boundary of the boundary is empty
%
% ----------------------------------------------------------------


\section{The boundary of the boundary is empty}


\subsection{Chain complexes}

We introduce the category of chain complexes, $\Chain$.

A {\it chain complex} $C$ is given by a sequence of vector spaces
${C_i}$ and linear maps $d_i:C_i\to C_{i-1}$ such that $d_{i-1}d_i=0$
for all $i$.

Here is a diagram:

\begin{center}
%\includegraphics[width=0.5\textwidth]{mypicture.png}
\includegraphics{chain.pdf}
\end{center}

The condition $d_{i-1}d_i=0$ is equivelant to
%requiring $\text{image}(d_i) \subseteq \text{kernel}(d_{i-1})$.
requiring the image of $d_i$ to be contained within the kernel of $d_{i-1}$:

\begin{center}
\includegraphics{figure_02.pdf}
\end{center}

Elements of the space $ B_i := \im(d_{i+1}) $ are known as {\it boundaries},
and elements of $ Z_i := \ker(d_i) $ are also known as {\it cycles}.

We form the quotient vector space
$\H_i(C) := B_i(C) / Z_i(C)$,
%$H_i := im(d_{i+1}) / kern(d_i)$,
called the i'th homology 
%vector space.
group (the group operation is given by the vector space addition.)
%We use a different font to dissambiguate the parity
%check matrix, which gets the regular font $H$.

The sequence of spaces ${\H_i}$ will also be denoted as
simply $\H.$ It can be taken to be a chain
complex with the zero boundary map.

We can always consider finite length chain complexes
by appending/prepending zero vector spaces and maps,
for example $A \to B \to C$ can be extended as

    $$ ... \to 0 \to A \to B \to C \to 0 \to ... $$


\subsection{Chain maps}

A chain map $f:C\to C'$ is a sequence of linear maps
$f_i:C_i\to C'_i$ that commute (intertwine) with the boundary map:
$f_{i-1}d_i = d'_if_i.$
Or in diagram form:

\begin{center}
\includegraphics{chainmap.pdf}
\end{center}

The main point about
a chain map is that it
induces a (linear) map of homology groups:

    $$\tilde{f}_i : \H_i\to\H'_i.$$



\subsection{The Hom functor}

%We consider the action of the boundary
%operator by pre-composition:

We consider the boundary operator acting
by pre-composition:

\begin{center}
\includegraphics{compose.pdf}
\end{center}

Given a chain complex $C$ and an arbitrary vector
space $V$ we see that the boundary
map $d_i$ acts on maps $f:C_{i-1}\to V$ to give a map $C_i\to V.$
We will fix $V$ to be the underlying field $\Z_2$ then
this action is ``multiplying on the right'',
ie. the transpose operation.
In this way we construct the dual cochain.


\begin{center}
\includegraphics{cochain.pdf}
\end{center}

This is the familiar covector construction.
In general, the so-called hom functor reverses the
direction of all the arrows (of some diagram.)
In our case this means simply that the transpose of a
product reverses the product: $(AB)^\top = B^\top A^\top.$

% ----------------------------------------------------------------
%
%
%
% ----------------------------------------------------------------


% Put this in the intro ??
%
%\subsection{Topologies give chain complexes}
%
%Homology theory in algebraic topology
%
%The boundary of a disc is a circle.
%The boundary of a circle is empty.
%A circle on a sphere is the boundary of a disc,
%but on a torus a circle may not always bound a disc.
%The homology of a space measures the failure of cycles to
%bound a higher dimensional object.
%
%% *** figure here ***
%
%There are many ways to carve up a space such that
%we can perform arithmetic (linear combinations) on
%finite dimensional objects within.
%Definition of n-dimensional object and it's
%(n-1)-dimensional boundary, such that these
%boundaries have trivial boundary themselves.
%
%Cubical (or simplicial) singular homology.
%\cite{Massey}.


\subsection{Classical linear codes}

A classical linear code may be specified as the
kernel of a parity check matrix $\S:\Z_2^n\to \Z_2^m.$
As a chain complex, this is the homology group at
$\Z_2^m.$

As a notational convenience we will sometime denote
a vector space by its (integer) dimension,
eg. $\S:n\to m.$

\setlength{\tabcolsep}{15pt}

\begin{center}
\begin{tabular}{ c c }
\underline{Chain}           &   \underline{Cochain}       \\[8pt]
%$\S:n\to m$      &      $\S^\top:m\to n$    \\[8pt]
\includegraphics{classchain.pdf}   &  \includegraphics{classcochain.pdf} \\[8pt]
$\H_1=\ker(\S)=:L$  &   $\H^0=\ker(\S^\top) =: L^\top $   \\[8pt]
$\H_0=m/{\im(\S)}=\coker(\S)$ &     $\H^1=n/{\im(\S^\top)}=\coker(\S^\top)$     \\[8pt]
\end{tabular}
\end{center}


\subsection{Quantum stabilizer codes}


A quantum (CSS) code is given by two parity
check matrices $\S_X:n\to m_X$ and $\S_Z:n\to m_Z,$
such that the chain condition $ \S_X \S_Z^\top = 0$
is satisfied.


\begin{center}
\begin{tabular}{ c c }
\underline{Chain}           &   \underline{Cochain}       \\[8pt]
\includegraphics{quchain.pdf}     &  \includegraphics{qucochain.pdf} \\[8pt]
$\H_2=\ker(\S_Z^\top)$                &   $\H^0=\ker(\S_X^\top) $   \\[8pt]
$\H_1=\ker(\S_X)/\im(\S_Z^\top)=:L_X$  &   $\H^1=\ker(\S_Z)/\im(\S_X^\top) =: L_Z $   \\[8pt]
$\H_0=m_X/{\im(\S_X)}=\coker(\S_X)$             &   $\H^2=m_Z/{\im(\S_Z)}=\coker(\S_Z)$     \\[8pt]
\end{tabular}
\end{center}

In the chain we are thinking of the space
$m_Z$ as the space of ``2-dimensional'' objects.
In the toric code this is the space of plaquettes,
but more generally we can think of these as ``generator
labels''. The space $n$ is associated with the physical
qubits, this is where the pauli operators reside.

\begin{center}
\includegraphics{toric_mZ.pdf}
\end{center}

In the toric code $n$ is the space of ``1-dimensional'' error
operators. The space $m_X$ is then the space of
(X-type) syndrome measurements. In the toric code
these are the ``zero-dimensional'' end-points of (Z-type)
error operators.

\begin{center}
\includegraphics{toric_mX.pdf}
\end{center}


% ----------------------------------------------------------------
%
%
%
% ----------------------------------------------------------------



\section{Tensor product}


The tensor product $C\otimes C'$ of two chain complexes $(C, d)$ and $(C', d')$
is given by

%    $$ (C\otimes C')_i = \bigoplus_{j+k=i} C_j\otimes C'_k $$
    $$ (C\otimes C')_i = \sum_{j+k=i} C_j\otimes C'_k $$

%This is a sum along the diagonals, for example:

with boundary map

    $$ d(c\otimes c') = d(c)\otimes c' + (-1)^{deg(c)}c\otimes d'(c').$$


%We now consider the following two dimensional diagram
%formed from two chain complexes $C$ and $C'$:

%We would like to form a product $C\otimes C'$ of two chain
%complexes $C$ and $C'$. To this end consider the following
%diagram:

To motivate these formulae, consider the following
two dimensional complex:

\begin{center}
\includegraphics{figure_03.pdf}
\end{center}

where $I$ indicates the appropriate identity map on each vector space.

To reduce this to a one dimensional structure we
(direct) sum along the diagonals, for example:

%Tensor product of two length three chain complexes
\begin{center}
\includegraphics{figure_04.pdf}
\end{center}

Now we add the arrows in an alternating fashion to get a boundary map.
The composition of two arrows in the same direction is
evidently zero, and
we use an alternating weight 
to force the two paths around each square to cancel:

\begin{center}
\includegraphics{figure_05.pdf}
\end{center}

With this definition of tensor product the category of
chain complexes becomes a monoidal category 
(See \cite{Baez09}, section 2.3 for a helpful discussion of
monoidal categories.)
%For the categorical definition of the tensor
%product see \cite{Baez09}, section 2.3.

\subsection{The Kunneth formula}

The homology group of the product inherits the same structure
as the underlying chain complex.
This is the import of the Kunneth formula:

    $$ \H_i(C\otimes C') = \sum_{j+k=i} \H_j(C) \otimes \H_k(C') $$

We now define a homomorphism
$ f:\H(C)\otimes \H(C') \to \H(C\otimes C')$
by its action on the subspaces: 
    $$ f:\H_j(C)\otimes \H_k(C') \to \H_{j+k}(C\otimes C').$$

defined by choosing $u_j\in \ker(d_j), u'_k\in \ker(d'_k)$
and then noting that 
    $$ (d_j\otimes I)(u_j\tensor u'_k) = 0,\ \ 
        (I\otimes d'_k)(u_j\tensor u'_k) = 0 $$
which means $u_j\tensor u'_k$ is in the kernel
of the tensor product boundary map, and so represents
an element of $\H_{j+k}(C\otimes C').$
Next check that the choice of $u_i, u'_j$ did not matter...

Weight of stabilizers...
Weight of logops...


\subsection{Product of two classical codes}


The hypergraph product of Tillich and Zemor \cite{Tillich09} is the
product of a classical code and the dual of a classical code.
We didn't define such a product above, but evidently if we
follow the arrows in the same way (or alternativy,
relabel the cochain) it should all work out.

\begin{center}
\includegraphics{hyperprod.pdf}
\end{center}

We use the Kunneth formulae to compute the logical
operators:

\begin{align*}
%    \H_1(C_1\otimes C_2^\top)   &= m_1/\im(\S_1)\otimes L_2^\top + L_1 \otimes n_2 / \im(\S_2^\top) \cr
    \H_1(C_1\otimes C_2^\top)   &= \coker(\S_1)\otimes L_2^\top + L_1 \otimes \coker(\S_2^\top) \cr
    |\H_1(C_1\otimes C_2^\top)| &= (m_1-|\im(\S_1)|)|L_2^\top| + |L_1|(n_2 - |\im(\S_2^\top)|) \cr
        &= (m_1-|\im(\S_1^\top)|)|L_2^\top| + |L_1|(n_2 - |\im(\S_2)|) \cr
        &= |\ker(\S_1^\top)||L_2^\top| + |L_1||\ker(\S_2)| \cr
        &= |L_1^\top||L_2^\top| + |L_1||L_2| \cr
\end{align*}

The toric code is obtained from the product of a
(classical) repitition code with its dual.
The important thing to note is that the parity check
matrix (stabilizer generators) is the object of primary
importance, not the space of logical operators.
To get the toric code we must start with a degenerate
parity check matrix  $
\S = \left(
\begin{array}{ccc}
1 & 1 & 0 \\
0 & 1 & 1 \\
1 & 0 & 1 \\
\end{array}
\right).$ 
Then $|L|=|L^\top|=1$ and we get the two logical
qubits in the product.
Using the matrix $
\S = \left(
\begin{array}{ccc}
1 & 1 & 0 \\
0 & 1 & 1 \\
\end{array}
\right).$ gives $|L|=1, |L^\top|=0$ which gives
one logical qubit in the product; this is a surface
code.

\subsection{Product of a classical and a quantum code}

This results in two separate quantum codes, as indicated in
the following diagram:

\begin{center}
\includegraphics{prodcq.pdf}
\end{center}

(There are also two classical codes at the endpoints.)

In this way we can generate the 3 (spatial) dimension toric code,
as a product of the 2D toric code with the repitition code.
Here we see the two resulting codes have sheets and lines for
logical operators, one code has x-type sheets and z-type lines, the
other code is x-type lines and z-type sheets.

\subsection{Product of two quantum codes}

Continueing this pattern we can generate three different
quantum codes as a product of two quantum codes:

\begin{center}
\includegraphics{prodqq.pdf}
\end{center}

For example, the product of the 2D toric code with itself
produces the 4D toric code with x and z type sheet operators,
thats the code in the middle.

Another example, the middle code of the product Stean x Stean has
parameters $[67, 1, 9].$


%\section{Code concatenation}

% ----------------------------------------------------------------
%
%
%
% ----------------------------------------------------------------


\section{Sums and Pushouts}

In any category the sum of two objects $A$ and $B$ is given by
an object $C$ and two maps $f:A\to C$ and $g:B\to C$. These
maps show how to embedd $A$ and $B$ into their ``sum''. A further
requirement is that $C$ is somehow minimal: any other contender
$C'$ for the sum of $A$ and $B$ with ``embedding'' maps $f'$ and $g'$
must factor uniquely through $C$.

In the category of sets we take the disjoint union of $A$ and $B$,
similarly in the category of topological spaces. The embedding
into the sum is the obvious inclusion map.

In the category of vector spaces, we take the direct sum $A\oplus B$, together
with maps $f=I\oplus 0$ and $g=0\oplus I.$
This extends to the category of chain complexes, which 
gives the disjoint union of two codes.

If we would like to join objects $A$ and $B$ along some
``sub-part'', $i:R\to A$, $j:R\to B$ we play the same game
but now require $f$ and $g$ to respect $i$ and $j$, that is,
$f\circ i = g\circ j.$ This is known as a ``pushout'' (of $i$ and $j$.)
In the category of sets, we would take the disjoint union as
before, and then identify those elements according to $(fi)(r) \sim (gj)(r), r\in R.$

This identification also works for topological spaces,
but for vector spaces we project out the {\it subspace} defined
by $fi - gj.$ The notation is then $A\oplus_R B.$

Extended to chain complexes we obtain a general way to
``weld'' two quantum codes together~\cite{Michnicki12}.

TODO...

\section{Non-CSS codes}

TODO...

%\section{Acknowledgements}
%Thanks to Gilles Zemor for some useful hints


%\section{Appendix: stabilizer vs vector space language}



%\todo{define stabilizer code}
%
%\todo{define toric code}


\chapter{Representations and Spectra of Gauge Code Hamiltonians}

\input{repr.tex}


\chapter{A Short Guide to Anyons and Modular Functors}

\input{guide.tex}


\chapter{Error Correction in a Non-Abelian Topologically Ordered System}


\section{Physical model}

The manifold underlying our system is a torus.
We endow this with a $L\times L$ square lattice of observables:
$$
    \Lambda := \bigl\{ \gamma_{ij} \bigr\}_{i,j=1,...,L}
$$
These observables are the physically accessible observables of
the noise reduction procedure we call the \emph{decoder.}
We call each such $\gamma_{ij}$ a \emph{tile.}
We show a small gap between the tiles but this is not meant
to reflect an actual physical gap.

The noise process acts to populate the manifold with
a randomly distributed set of pair creation processes,
whose size is much smaller than the resolution of the lattice.
We model this as a random distribution of pair-of-pants:
\begin{center}
\includegraphics[width=0.3\columnwidth ]{pic-pair-create.pdf}
\end{center}

Each such pair will have vacuum total charge and so the observables
$\gamma_{ij}$ will only see pairs that intersect, ie. we
need only consider distributing these pairs
transversally along edges of the tiles.

In order to compute measurement outcomes for the $\gamma_{ij}$
we first need to concatenate any two curve diagrams that 
participate in the same $\gamma_{ij}.$
Because each curve has vacuum total charge this can be
done in an arbitrary way:
\begin{center}
\includegraphics[width=0.3\columnwidth ]{pic-join-pairs.pdf}
\end{center}

Working in the basis picked out by the resulting curve
diagrams, we can calculate measurement outcomes for each tile,
the result of which is recorded on the original curve:
\begin{center}
\includegraphics[width=0.3\columnwidth ]{pic-curve-uniq.pdf}
\end{center}
%\cggb{It might be helpful to be a bit more explicit either here or later about exactly how we perform this step, moving charges around with the paperclip algorithm until they are all neighbouring and then we are in a standard basis and can use F-moves to calculate fusion outcomes.}
%\simon{good idea}


%The details of how this is implemented computationally,
%the data structures and algorithm, we have not yet described
%and this is what we turn to next.

%
% ~~~~~~~~~~~~~~~~~~~~~~~~~~~~~~~~~~~~~~~~~~~~~~~~~~~~~~~~~~~~~~~~~~~~~~~~~~~~
%
\section{Combinatorial curve diagrams}

The basic data structure involved in the
simulation we term a \emph{combinatorial curve diagram.}
% XXX define "piece of curve"
Firstly, we will require each curve to intersect 
the edges of tiles transversally,
and in particular a curve will not touch a tile corner.

For each tile in the lattice,
we store a combinatorial
description of the curve(s) intersected with that tile.
Each component of such an intersection we call a \emph{piece-of-curve.}
\begin{center}
\includegraphics[]{pic-cells.pdf}
\end{center}

We follow essentially the same approach as taken in \cite{Abramsky2007} 
to describe elements of a Temperley-Leib algebra, but
with some extra decorations.
The key idea is to store a \emph{word} for each tile, comprised of
the letters $\bigl<$ and $\bigr>$.
%The encoding works as follows.
Reading in a clockwise direction around the edge of
the tile from the top-left corner,
we record our encounters with each piece-of-curve,
writing~$\bigl<$ for the first encounter, and~$\bigr>$ for the
second.
We may also encounter a dangling piece-of-curve
(the head or the tail), so we use another symbol $*$ for this.
The words for the above two tiles will then be 
$\bigl<\bigl<\bigr>\bigr>\bigl<\bigr>$ and $\bigl<\bigr>*\bigl<\bigl<\bigr>\bigr>.$
When the brackets are balanced,
each such word will correspond one-to-one with an intersection
of a curve in a tile, up to a continuous deformation of the interior of the tile.
Ie. the data structure 
will be insensitive to any continuous deformation of the interior of the tile,
but the simulation does not need to track any of these degrees of freedom.

\begin{center}
\includegraphics[]{pic-cells-0.pdf}
\end{center}

We will also need to record
various other attributes of these curves,
and to do this we make this notation more elaborate
in the paragraphs {\bf (I)}, {\bf(II)} and {\bf(III)} below.
Each symbol in the word describes an intersection of
the curve with the tile boundary,
and so as we decorate these symbols these decorations will
apply to such intersection points.

\begin{center}
\includegraphics[]{pic-cells-1.pdf}
\end{center}

{\bf (I)} We will record the direction of each piece-of-curve,
this will be either an {\tt in} or {\tt out} decoration for each symbol.
Such decorations need to balance according to the brackets.
The decorated symbols $*_{\mbox{\tt in}}$ and 
$*_{\mbox{\tt out}}$ 
will denote respectively either
%the head, $c(1)$ or the tail $c(0)$ of a curve.
the head or the tail of a curve.
The words for the diagrams above now read as
$ \bigl<_{\mbox{\tt in}}\bigl<_{\mbox{\tt out}}\bigr>_{\mbox{\tt in}}
    \bigr>_{\mbox{\tt out}}\bigl<_{\mbox{\tt out}}\bigr>_{\mbox{\tt in}}$
and
$ \bigl<_{\mbox{\tt in}}\bigl>_{\mbox{\tt out}}*_{\mbox{\tt in}}
    \bigr<_{\mbox{\tt out}}
    \bigr<_{\mbox{\tt in}}\bigl>_{\mbox{\tt out}}\bigr>_{\mbox{\tt in}}.
$

{\bf (II)} We will record,
for each intersection with the tile edge, 
a numeral indicating which of the four
sides of the tile the
intersection occurs on.
Numbering these clockwise from the top as $1, 2, 3, 4$ we have for the above curves: 
$\bigl<_1\bigl<_2\bigr>_2\bigr>_3\bigl<_3\bigr>_4$ 
and $\bigl<_1\bigr>_1*_2\bigl<_3\bigl<_3\bigr>_4\bigr>_4.$

{\bf (III)} Finally, we will also decorate these symbols with anyons.
This will be an index to a leaf of a (sum of) fusion tree(s).
This means that anyons only reside on the curve close
to the tile boundary,
and so we cannot have more than two anyons
for each piece-of-curve. 
The number of such pieces is arbitrary, and so this
is no restriction on generality.

\begin{center}
\includegraphics[]{pic-cells-2.pdf}
\end{center}

In joining tiles together to make a tiling we will
require adjacent tiles to agree on their shared boundary.
This will entail sequentially pairing symbols in the
words for adjacent tiles
and requiring that 
the {\tt in} and {\tt out} decorations are matched.
Because the word for a tile proceeds conter-clockwise
around the tile, this pairing will always reverse the
sequential order of the symbols of adjacent tiles.
For example, given the above two tiles we sequentially pair the 
$\bigl<_{\mbox{\tt out},2}\bigr>_{\mbox{\tt in},2}$ 
and $\bigr>_{\mbox{\tt out},4}\bigr>_{\mbox{\tt in},4}$
symbols with opposite order so that
$\bigl<_{\mbox{\tt out},2}\sim\bigr>_{\mbox{\tt in},4}$
and $\bigr>_{\mbox{\tt in},2} \sim \bigr>_{\mbox{\tt out},4}.$ 

%Two other consistency relations are enforced on such a combinatorial curve diagram:
%we require adjacent tiles to agree on their boundaries, 
%and every curve diagram
%must have two ends.
%One final consistency
%relation is enforced by requiring 
%We require every curve diagram to have two ends, this means
%that there are no loops.

Note that in general this data structure will store many disjoint curve diagrams
$c_i:[0,1]\to D_{n_i}$ within a disc $D_m$ where $\sum n_i = m.$

%For a given piece-of-curve, we can subtract the numeral with
%the {\tt in} label from the numeral with the {\tt out} label
%to get an integer in $\{ \}$ WRONG
%We can also just use the numerals of the boundary edges
%that the piece-of-curve intersects with, subtracting the
%``exit'' boundary numeral from the ``entry'' boundary numeral. WRONG

For each piece-of-curve, apart from a head or tail, there is an associated 
number we call the \emph{turn number}. This counts the number
of ``right-hand turns'' the piece-of-curve makes as it
traverses the tile, with a ``left-hand turn'' counting as $-1.$
(To be more rigorous, we would define this number using the
winding number of the simple closed curve formed by the
piece-of-curve adjoined to a segment of the boundary of the tile 
traversed in a clockwise direction.)
This number will take one of the values $-2, -1, 0, 1, 2:$
\begin{center}
\includegraphics[]{pic-cells-3.pdf}
\end{center}


%Two disjoint curves can be joined by...
%
%Curves can be simplified along sections without any anyons, ...

% Each piece will have +1, +2, 0, -1, -2 right-hand turns...

% mention Jones' planar algebras ?

%
% ~~~~~~~~~~~~~~~~~~~~~~~~~~~~~~~~~~~~~~~~~~~~~~~~~~~~~~~~~~~~~~~~~~~~~~~~~~~~
%

\section{The paperclip algorithm}

In the description of the refactoring theorem
above we thought of $R$-moves as acting on the basis of
the system as in the Heisenberg picture.
Now we switch to an equivelant perspective and
consider $R$-moves as transport of anyon charges
as in a Schrodinger picture.
The anyons will be transported around the lattice
by moving them along tile edges.
%\simon{Note that transport here is the same as the refactoring from above.}
In general, such a transport will intersect with a
curve diagram in many places.
Each such intersection is transverse,
and we use each intersection point to cut
the entire transport into smaller paths each of
which touch the curve diagram twice.
%Each intersection with a curve diagram
%will then be transverse, and we
%decompose the entire path into a sequence of
%paths each of which 
%join consecutive intersections.
%Transport of an anyon can be decomposed into
%moves between adjacent components of a curve
%diagram.
The origin and destination of such an anyon path
now splits the curve diagram $c:[0, 1]\to D_n$ 
into three disjoint pieces which we term
\emph{head}, \emph{body} and \emph{tail}, where
the head contains the point $c(1)$, the tail
contains $c(0)$ and the body is the third piece.
These arise with various arrangements, but here
we focus on one instructive case, the
other cases are similar:
%We look at the particular case of moving along
%one edge of a tile,
transporting along one edge of a tile \emph{forwards} 
(from tail to head) along a curve diagram:
\begin{center}
\includegraphics[]{pic-move-anyon.pdf}
\end{center}

This arrangement is equivalent (isotopic) to one of four 
``paperclips'', which we distinguish between by counting how
many \emph{right-hand turns} are made along the body of the curve diagram.
We also show an equivalent (isotopic) picture where the
curve diagram has been straightened, and the resulting distortion
in the anyon path:
\begin{center}
\includegraphics[]{pic-paperclip.pdf}
\end{center}
The sequence of anyons along the head, body and tail, we denote as $H, B$ and $T,$
respectively.
These sequences have the same order as the underlying curve diagram, and 
we use
$H^r, B^r$ and $T^r$ to denote the same anyons with the reversed order.
Using the above diagram, we can now read off the $R$-moves for each
of the four paperclips:
\begin{align*}
-2:&\ R[B] \\
-4:&\ R[H^r]\ R[H]\ R[B] \\
+4:&\ R[B]\ R[T]\ R[T^r] \\
+6:&\ R[H^r]\ R[H]\ R[B]\ R[T]\ R[T^r] \\
\end{align*}
where notation such as $R[B]$ is understood as sequentially clockwise braiding around
each anyon in $B$.

That these four paperclips exhaust all possibilities can be seen by
considering the winding number of the simple closed curve made
by combining the body of the curve diagram with the path followed by
the anyon (appropriately reversing direction as needed).

%
% ~~~~~~~~~~~~~~~~~~~~~~~~~~~~~~~~~~~~~~~~~~~~~~~~~~~~~~~~~~~~~~~~~~~~~~~~~~~~
%

\section{Decoding algorithm}

After the noise process is applied to the system,
the error correction proceeds as a dialogue between the
decoder and the system. 
The decoder measures succesively larger and larger
regions of the lattice
until there are no more charges 
or a topologically non-trivial operation has occured
(an operation that spans the entire lattice.)
Here we show this in a process diagram, with time running up
the page:
\begin{center}
\includegraphics[]{pic-process.pdf}
\end{center}

So far we have discussed the simulation of the (quantum) system
and now we turn to the decoder algorithm.
Here is a pseudo-code listing for this,
and we explain each step via an example below.

\begin{verbatim}
 1:  def decode():
 2:      syndrome = get_syndrome()
 3:      
 4:      # build a cluster for each charge
 5:      clusters = [Cluster(charge) for charge in syndrome]
 6:  
 7:      # join any neighbouring clusters
 8:      join(clusters, 1)
 9:      
10:      while clusters:
11:      
12:          # find total charge on each cluster
13:          for cluster in clusters:
14:              fuse_cluster(cluster)
15:      
16:          # discard vacuum clusters
17:          clusters = [cluster for cluster in clusters if non_vacuum(cluster)]
18:      
19:          # grow each cluster by 1 unit
20:          for cluster in clusters:
21:              grow_cluster(cluster, 1)
22:      
23:          # join any intersecting clusters
24:          join(clusters, 0)
25:  
26:      # success !
27:      return True
\end{verbatim} % see decode.py

First, we show the result of the initial call to {\tt get\_syndrome()}, on line 2.
The locations of anyon charges are highlighted in red.
For each of these charges we build a {\tt Cluster}, on line 5.
Each cluster is shown as a gray shaded area.
\begin{center}
\includegraphics[]{pic-decode-0.pdf}
\end{center}
The next step is the call to {\tt join(clusters, 1)}, on line 8,
which joins clusters that are separated by at most one lattice
spacing. We now have seven clusters:
\begin{center}
\includegraphics[]{pic-decode-1.pdf}
\end{center}
Each cluster is structured as a rooted tree, as indicated by
the arrows which point in the direction from the leaves to
the root of the tree. 
This tree structure is used in the call to {\tt fuse\_cluster()},
on line 14.
This moves anyons in the tree along the arrows to the root, 
fusing with the charge at the root.
\begin{center}
\includegraphics[]{pic-decode-2.pdf}
\end{center}
For each cluster, the resulting charge at the root is taken as the charge of
that cluster. Any cluster with vacuum total charge is then discarded (line 17).
%In our example, we assume all these charges are non-vacuum.
In our example, we find two clusters with vacuum charge and we discard these.
The next step is to grow the remaining clusters by one lattice spacing (line 20-21),
and join (merge) any overlapping clusters (line 24).
\begin{center}
\includegraphics[]{pic-decode-3.pdf}
\end{center}
%Now we are down to two clusters.
Note that we can choose the root of each cluster arbitrarily,
as we are only interested in the total charge of each cluster.

We repeat these steps of fusing, growing and then joining clusters (lines 10-24.)
If at any point this causes a topologically 
non-trivial operation, the simulation aborts and a failure
to decode is recorded.
Otherwise we eventually run out
of non-vacuum clusters, and the decoder succeeds (line 27).
Note that for simplicity we have neglected the boundary of the lattice in
this example.

%\cggb{Maybe it is worthwhile to briefly recall the broad structure of our simulation somewhere here to help structure the discussion. I.e.~we have first noise creation, then we iterate \{syndrome measurement, classical decoding algorithm, transport\} until failure or success.}
%\simon{I agree the structure needs work.}

\section{Computation of homologically non-trivial operators}\label{s:homnontrivial}

Specializing to the Fibonacci case,
we write the non-trivial $F$-moves as the following
skein relations:
\begin{align*}
\includegraphics[]{pic-skein1.pdf}
\end{align*}

The sollid lines represent Fibonacci world-lines.
The dotted lines represent vacuum charges,
and we are free to include these lines or not.
We leave these anyon paths
as undirected because Fibonacci anyons are
self-inverse.
The non-trivial $R$-moves are:
\begin{align*}
\includegraphics[]{pic-skein2.pdf}
\end{align*}

Removing bubbles:
\begin{align*}
\includegraphics[]{pic-bubble.pdf}
\end{align*}

Here we show a process where a 
Fibonacci anyon travels around the torus and
anihilates itself. Twice.
The vertical lines represent a periodic
identification.
\begin{align*}
\includegraphics[]{pic-logops.pdf}
\end{align*}

The state is not normalized.
Also involves post-selection, as there is
another process that involves leakage...
The first equation is an $F$-move, 
the second equation is a translation in the
horizontal direction, and the last equation
follows from the rule for collapsing bubbles.

Continuing in this way, we compute the $k$-fold
logical operator:
\begin{align*}
\includegraphics[]{pic-kfold.pdf}
\end{align*}

where $f_k$ is the $k$-th element of the Fibonacci
sequence $\{1, 1, 2, 3...\}.$




%\appendix
%\chapter{First Appendix}

\bibliography{refs}{}
\bibliographystyle{abbrv}



\end{document}
